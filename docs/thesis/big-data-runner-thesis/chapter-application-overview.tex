\chapter{Badania}
Podczas badań zostały przeprowadzone testy wydajności dla trzech operacji, które są reprezentowane przez końcowy interfejs użytkownika. Dokładny opis znajduje się w sekcji \ref{sec:user_interfaces}. Na potrzeby badań operacje nazwijmy następująco:
\begin{itemize}
	\item {Word Count}
	\item {Filter}
	\item {Reject}
\end{itemize}
Testy wydajności zostały wykonane na danych zebranych poprzez Twitter Developer API opisanym w sekcji \ref{sec:twitter-api}. Z racji iż strumieniowanie dla platformy Spark zapisuje dane cyklicznie, powstał katalog w systemie HDFS zawierający paczki danych zapisanych co 360 sekund. Na potrzeby badań wszystkie dane pobrane ze zdalnego API zostały połączone w jeden plik o wielkości \textbf{4GB}, który posłużył za źródło danych podczas wykonywania testów wydajności. Badania zostały przeprowadzone na maszynie o specyfikacji:
\begin{itemize}
	\item Procesor: Intel Core i7-4702MQ (2.2GHz Quad-core + HT) 
	\item Pamięć RAM: 16GB
	\item Pamięć masowa: 1 TB 5400rpm\footnote{Revolutions per minute - obroty na minutę} SATA
	\item Karta graficzna: NVIDIA GeForce GT 750M (2G)  
\end{itemize}
Podczas badań specyfikacja karty graficznej nie ma wpływu na wyniki, gdyż dla badanych operacji wykorzystywane są trzy jednostki obliczeniowe: procesor, pamięć RAM, dysk twardy.
\newline Pomiary wydajności zostały dokonane przy pomocy programu \textbf{cURL}.\footnote{\url{https://curl.haxx.se/}} Testy obejmują siedem następujących czynników:
\begin{itemize}
	\item{Czas rozwiązywania adresu\footnote{time\_namelookup}}
	\item{Czas połączenia TCP\footnote{time\_connect}}
	\item{Czas wymiany informacji między połączeniami (handshake)\footnote{time\_appconnect}} 
	\item{Czas całkowity przed wymianą informacji\footnote{time\_pretransfer}}
	\item{Czas przekierowań\footnote{time\_redirect}}
	\item{Czas przed którym pierwszy bajt danych miał zostać wysłany\footnote{time\_starttransfer}}
	\item{Czas całkowity\footnote{time\_total}}
\end{itemize}
Dla operacji \textit{Word Count} wyniki wydajności są przedstawione w tabeli \ref{tab:word-count-results}. 
\begin{table}[]
	\centering
	\caption{Wyniki wydajności dla zliczania ilości wystąpień poszczególnych fraz tekstowych (oddzielonych spacją) znalezionych w źródle danych oraz zapisu wyniku do pliku}
	\label{tab:word-count-results}
	\begin{tabular}{|l|l|l|}
		\hline
		Czas w sekundach    & Hadoop     & Spark      \\ \hline
		time\_namelookup    & 0.000020   & 0.000023   \\ \hline
		time\_connect       & 0.000097   & 0.000116   \\ \hline
		time\_appconnect    & 0.000000   & 0.000000   \\ \hline
		time\_pretransfer   & 0.000116   & 0.000134   \\ \hline
		time\_redirect      & 0.000000   & 0.000000   \\ \hline
		time\_starttransfer & 402.792577 & 111.232410 \\ \hline
		time\_total         & 402.792625 & 111.232446 \\ \hline
	\end{tabular}
\end{table}
Dla operacji \textit{Filter} wyniki wydajności są przedstawione w tabeli \ref{tab:filter-results}.
\begin{table}[]
	\centering
	\caption{Wyniki wydajności dla zliczania ilości linii zawierających frazę tekstową zdefiniowaną przez użytkownika}
	\label{tab:filter-results}
	\begin{tabular}{|l|l|l|}
		\hline
		Czas w sekundach    & Hadoop    & Spark     \\ \hline
		time\_namelookup    & 0.000022  & 0.000031  \\ \hline
		time\_connect       & 0.000090  & 0.000120  \\ \hline
		time\_appconnect    & 0.000000  & 0.000000  \\ \hline
		time\_pretransfer   & 0.000110  & 0.000145  \\ \hline
		time\_redirect      & 0.000000  & 0.000000  \\ \hline
		time\_starttransfer & 42.685888 & 80.167355 \\ \hline
		time\_total         & 42.685924 & 80.167405 \\ \hline
	\end{tabular}
\end{table}
Dla operacji \textit{Reject} wyniki wydajności są przedstawione w tabeli \ref{tab:reject-results}.
\begin{table}[]
	\centering
	\caption{Wyniki wydajności odrzucenia linii zawierających frazę tekstową zdefiniowaną przez użytkownika oraz zapisania wyniku do pliku}
	\label{tab:reject-results}
	\begin{tabular}{|l|l|l|}
		\hline
		Czas w sekundach    & Hadoop     & Spark     \\ \hline
		time\_namelookup    & 0.000025   & 0.000049  \\ \hline
		time\_connect       & 0.000098   & 0.000184  \\ \hline
		time\_appconnect    & 0.000000   & 0.000000  \\ \hline
		time\_pretransfer   & 0.000118   & 0.000237  \\ \hline
		time\_redirect      & 0.000000   & 0.000000  \\ \hline
		time\_starttransfer & 238.408629 & 45.186254 \\ \hline
		time\_total         & 238.408673 & 45.186296 \\ \hline
	\end{tabular}
\end{table} 
\todo{przeprowadzone badania - może inny tytuł rozdziału?}